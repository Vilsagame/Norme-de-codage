%%%%%%%%%%%%%%%%%%%%%%%%%%%%%%%%%%%%%%%%%%%%%%%%%%%%%%
%% \file presentation.tex
%% \version 0.1
%% \author Vilsafur
%%
%% Présentation du document
%%
%%%%%%%%%%%%%%%%%%%%%%%%%%%%%%%%%%%%%%%%%%%%%%%%%%%%%%

\begin{abstract}
Les normes de codages sont plus des codes de bonnes conduites que des règles stricts et figées. Il faut bien comprendre qu'elles ne sont pas là pour handicaper le développeur mais bien au contraire à l'avantager lors de ces différentes phases de relecture, que ce soit pour déboguer un code ou encore améliorer ce dernier. \\
De plus, elles permettent au futurs contributeurs de mieux appréhender le code et de mieux le comprendre.\\
~\\
Un code propre et bien indenté est toujours plus facile à lire et à comprendre.\\
Ces différentes normes seront donc à appliquer dans l'ensemble des projets mis en place au nom de Vilsagame.
\end{abstract}