%%%%%%%%%%%%%%%%%%%%%%%%%%%%%%%%%%%%%%%%%%%%%%%%%%%%%%
%% \file html_css.tex
%% \version 0.1
%% \author Vilsafur
%%
%% Définition des règles spécifiques aux langages
%% HTML et CSS
%%
%%%%%%%%%%%%%%%%%%%%%%%%%%%%%%%%%%%%%%%%%%%%%%%%%%%%%%

\section{Règles spécifique aux langages HTML5 et CSS3}

\subsection{CSS}
Nous utiliserons un détournement de la méthode \href{http://getbem.com/}{BEM} décrite dans l'article de tarh sur developpez.com : \url{http://tarh.developpez.com/articles/2014/bonnes-pratiques-en-css-bem-et-oocss/}.\\
En voici les principaux axes :
\begin{itemize}[label=\textbullet]
  \item Nommage des classes :
  \begin{itemize}
    \item ComponentName
    \item ComponentName.modifierName
    \item ComponentName-descendantName
    \item ComponentName-descendantName.modifierName
    \item ComponentName.isStateOfComponent
  \end{itemize}
  \item Utilisation de classes "transversales".
  \item Un composant ne contient aucune information de positionnement.
  \item Ne pas utiliser une classe CSS pour une ancre JavaScript, préférer une classe nommée ".js-".
\end{itemize}

\subsection{SASS}
Afin de faciliter l'écriture des fichiers CSS, nous utiliserons le préprocésseur SASS avec le language SCSS.\\
De plus, nous utiliserons les normes suivantes :
\begin{itemize}[label=\textbullet]
  \item Les variables seront sous la forme : element-propriété
  \item Les mixins et les fonctions seront nommés comme vue précédemment (cf \ref{nommageFonctions})
\end{itemize}
